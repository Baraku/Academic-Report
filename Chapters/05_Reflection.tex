\section{Reflection}

\subsection{Difficulties}
\label{Difficulties}
During the undertaken of the project, the student was faced with a number of difficulties which are described in detail in the following sections.

\subsubsection{Design Thinking}
One of the cornerstones of using iCIMS is a fundamental change in design thinking. Throughout university, student are taught a design approach that consists of requirements gathering, concept creation, development and testing. These are the main aspects of any design methodology and each is the focus at given point in any methodology. Designing with iCIMS changes that. iCIMS, in essence, melds all of these concepts into a single concept. While the user is designing he or she is also testing their design thinking and modifies the design on the fly based on requirements and needs. It was difficult to truly turn away from standard software practices and undertake ``pure" designing by just creating forms and changing them based on feedback. Several times during the project, the student would catch themselves thinking about prototyping the forms or creating mockups or even how they would create said form \emph{the old fashioned way}, through programming. This change in thinking caused some uneasiness or even doubt at times. 

\subsubsection{The Golden Middle}
After having gathered the requirements for the project, it became evident that different stakeholders had different needs. These sometimes opposed each other forcing the student to attempt to find the golden middle, a compromise that would satisfy both sides. A very good example of this is the fact that senior staff wanted a handover by exception but junior staff required a handover that contained all of the information. These are two opposing views for which an adequate solution must be found. Finding that solution is not always easy and proved a challenge in some cases. This was also applicable to situations surrounding the project itself where hospital stakeholder views were at odds with academic views. These kinds of issues are especially challenging for a student because they require  a great deal of attention in order to solve successfully.

\subsubsection{iCIMS}
One of the major difficulties encountered during the project was the fact that the student was pushing iCIMS as far as it would go and was thus exposed to major growing pains of the tool. Several times during the semester, the student was faced with limitations on the part of the tool that required rethinking a design often causing a considerable loss of time. The growing pains of the tool also exposed numerous issues and bugs that slowed the student down and sometimes even brought the work to a grinding halt. Major issues included the tool's performance when viewing forms in the web browser as well as features not working as intended or even breaking as a result of bug fixes undertaken by the development team.  

\subsubsection{Time}
Another issue rather than difficulty that arose during the project was the issue of time. Because of the type of work the nurses were doing during their shifts, they had little time to spare. Meetings could not be planned ahead of time and were undertaken in an ad-hoc manner; going up to the ward and seeing if the nurse was available. It is quite understandable that the nurses had little time but it nevertheless made the project more difficult. Time was also somewhat of an issue on the student side. It was not possible for the student to go to the hospital every day during the week because of other academic obligations. This limited the time the student had to accomplish tasks at the hospital.

\subsection{Lessons Learned}
Industry projects lend themselves very well to students learning valuable lessons for their future endeavours. This project was no different. Something that stood out especially was the fact that communication is key. The student had undertaken an industry project in a prior degree that also put emphasis on communication. Communication is key because both the student as well as the SAN staff come from two very different ``worlds" but must work together to produce a successful outcome. 
\\ \\
Another lesson learned is the fact that nothing is free. The student had to invest time and effort into introducing himself to staff and presenting the objectives of the project. Only after clinical staff were somewhat familiar with the student could the real design work begin because the staff became more comfortable around the student and were more willing to offer their time and feedback. The student was fortunate enough to guided by a very enthusiastic nurse at times that not only showed him around but also help introducing him to the staff. These social relationships play a vital role in the success of a project.
\\ \\
A further valuable lesson learned is the fact that nothing great ever comes easy. Even though there were many difficulties with iCIMS and a great deal of time problems, the project was still successful and shows promise for the future. Having to deal with a domain as complex and information rich as a hospital has shown how much computer information systems can do to support and improve the work of staff. It is a stoney road that leads to a digital clinical handover but it is a road worth taking.

\subsection{Future Suggestions}
In the future, showing more presence while working at a client's site would allow for a better interaction with client users. By interacting more with the staff on a regular basis, it would form a bond that could be used to more effectively enlist the help of the users in designing a solution that meets their needs and requirements. Improving soft skills is vital to any successful undertaking and would thus be the focus of future improvements.