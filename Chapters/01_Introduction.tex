\section{Introduction}
\subsection{Client Profile}
Originally opened in Wahroonga on January 1 1903 as a 70 bed Sanitarium, the Sydney Adventist Hospital (SAH), known to the local residents as `The San', is a not-for-profit hospital of the South Pacific Division of the Seventh-day Adventist Church.  Today, the hospital is a private hospital offering acute care and currently has 358 licensed overnight beds. SAH is the largest single campus private hospital within NSW and was the first of its kind to be accredited by the Australian Council on Healthcare Standards. SAH is proud to have won the Australian Private Hospitals Association Award for Clinical Excellence in the category 70 beds and over in 2006.
 \\ \\
The San prides itself on being the single biggest employer within the Hornsby-Kuring-gai area employing over 2,200 staff and around 700 accredited medical practitioners. Together, the SAN staff care for more than 50,000 inpatients and about 160,000 outpatients. The San is also known for its maternity wards and is proud to be bringing over 2,000 babies a year into the world. The SAN, being one of few private hospitals to offer emergency care, admits over 20,000 patients annually making it NSW's largest and busiest emergency care department among private hospitals. The SAN offers medical services ranging from acute surgical, medical and obstetric care to complex cardiac and orthopaedic procedures. The SAN boasts cutting edge facilities that include a dozen operation theatre suites, 3 state-of-the-art Cardiac Catheterisation Laboratories and Australia's first dual source CT scanner. The SAN is also responsible for operating the San Day Surgery Hornsby and Dalcross Adventist Hospital, located in Killara.
\\ \\
With the mission statement ``Christianity in Action", the SAN not only offers world class care to the patients within the hospital, but also to disadvantaged third world men, women and children as part of its HealthCare Outreach program. Since its inception in 1986, the HealthCare Outreach program has undertaken 100 trips to 13 different countries culminating in over 2,800 surgeries and lives saved. 
\newpage

\subsection{Project Description and Scope}
\subsubsection{Project Description}
The SAN Hospital Information System has to service many different clinical specialities and environments. This project will develop a prototype application in the form of a simulator of a novel HIT system for surgical patients. These patients typically have specific and predictable post-surgical outcomes and hospitalisation time-frames, as outlined in various surgical clinical pathways (e.g. Urology such as, Greenlight Laser Prostatectomy, Ear, Nose and Throat (ENT) and Plastics). Caring for these patients requires information systems for multidisciplinary nursing and allied health staff.
Constructing a requirements document will be a complex task but give students the richest possible experience in understanding all the stages of requirements gathering, systems design and systems implementation. The project will use a research technology simulator that enables the process of requirements gathering and system design to be integrated as a single process and thereby enable validation of requirements by their implementation into a design simulator. 

\subsubsection{Scope}
The project will commence with the gathering of requirements by meeting and interviewing various clinical staff fulfilling a variety of roles on the surgical ward, level 11. The student will also gather all paper based forms in use on the ward as references during the design process. Upon completion of the first phase of requirements gathering, a requirements document will be created and will represent the basis for design decisions. Throughout the rest of the semester the student will update the requirements document as necessary. The student will design forms, including the clinical handover, in the simulator as well as obtain end user feedback during the majority of the semester. Towards the end of the semester, the student will undertake user acceptance testing as well as evaluations of the work done. The project will conclude with a first draft of the clinical handover form and a presentation to SAN staff. The project will finish at the end of the academic semester.

\newpage
\subsection{Project Objectives}
\label{Project Objectives}
\begin{itemize}
\item Collect the requirements for a Clinical Handover for use by nurses, allied health and medical staff in the care of surgical patients
\item Produce an accurate record of the information each worker needs access to in the form of a requirements document including process flows
\item Design and develop a prototype which simulates an electronic clinical information system with handover processes for nurses, doctors and other clinical staff
\end{itemize}

\subsubsection{Risks}

\begin{tabular}{|l|l|l|l|l|}
\hline
{\bf Ref \#} & {\bf Probability} & {\bf Impact} & {\bf Description} & {\bf Mitigation} \\
\hline
R.1 & High & Medium & Reduced performance through & Increase allotted tool  \\ & & & use of new technology & usage time  \\ 
\hline
R.2 & Low & High & Unable to complete project  & frequent comm- \\ & & & objectives due to simulator & unication with \\ & & & issues & simulator developers  \\  
\hline
R.3 & Medium & Medium & Scope creep & Clearly outline scope \\ & & & &  at outset of project \\
\hline
\end{tabular}

\subsubsection{Assumptions}

\begin{tabular}{|l|l|l|}
\hline
{\bf Ref \#} & {\bf Description} \\
\hline
A.1 & The simulator will not need to connect to existing SAN applications \\
\hline
A.2 & We will have access to a simulator developer \\
\hline
A.3 & A project manager will be available to assist us in our work at the hospital \\ 
\hline
A.4 & We are not developing a system for actual use \\
\hline
\end{tabular}

\subsubsection{Issues}

\begin{tabular}{|l|l|l|l|}
\hline
{\bf Ref \#} & {\bf Priority} & {\bf Description} & {\bf Owner} \\
\hline
I.1 & High & Simulator bugs \& issues & Simulator Developer \\
\hline
I.2 & Medium & Exposed to immense amount of information & Student \\
\hline
I.3 & Medium & Sporadic staff availability  & Student \& PM \\
\hline
I.4 & Low & Time constraints due to university courses & Student \\
\hline
\end{tabular}


\newpage
\subsection{Anticipated Outcomes / Results for the Project}
\subsubsection{First Draft Computerised Handover Form}
At the conclusion of this project, the student should have designed a first draft of a computerised handover form. This draft does not need to contain all information required but should focus on the most important pieces of information especially in regards to nurses. It should convey all relevant design decisions and be capable of representing patients with varying degrees of complexity, the degree to which a patient is ill.

\subsubsection{Understanding of IT/IS within the Health Domain}
The student will have gained a general understanding of not only the health domain but also the role of IT IS systems within a hospital setting. The student should see the advantages of using computerised information systems to support the clinical staff in their daily work.
 
 \subsubsection{Requirements Analysis Complete}
By the end of the semester, the student should have completed the requirements analysis for the project. This should include all relevant requirements up to the end of the academic semester. As part of the requirements analysis, the student should have created as-is and to-be process diagrams for the nurse to nurse handover.

\subsubsection{Design of a Simulation of the Desired System}
A designer tool was to be used to create a simulation of the information system that conformed to the collected requirements. The simulation will have been assessed by clinical staff for its conformance to the expressed requirements.

\subsection{Benefits of the Project}
\subsubsection{Technology Evangelisation}
Through the project, the student will be able to show the advantages and abilities of computerised information systems to clinical staff, in particular nursing staff. Although the clinical staff at the SAN are using applications to record information already, not all aspects of their daily work are digitised including handover. By working with staff throughout the semester, the student will be able to generate end-user buy-in and support for a computerised clinical handover.

\newpage
\subsubsection{Pilot Project}
This project constitutes a pilot project in the sense that future students can build upon the achievements of this project. The foundation, both in regard to system design as well as end-user exposure, will provide future students with a lower entry barrier into the clinical domain and into the SAN.

\subsubsection{End-User Driven Development}

The project will expose the student to a new development methodology model in conjunction with a design tool. This will provide the student with the opportunity to compare and contrast traditional development methodologies with this newly created model. 

\subsubsection{Trialing the Designer Tool}
This project allowed the student to use a health information system application called \gls{iCIMS} which is intended for creating a simulator of the ideal system for the user. The project can thus be seen as a trial for the designer tool enabling the student to not only use a new technology but also to document bugs and suggest enhancements to the system. In essence, the student ran the application through its paces and noted shortcomings as well as possible additional features allowing the application to go through a maturation process. 