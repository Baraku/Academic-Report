\section{System Design}
This chapter will describe SANSURGIMS, ``SAN Surgical Information Management System", which embodies the results of the project. The overarching goal of the project was to design a computer based handover prototype for clinical staff as outlined in \ref{Project Objectives}. The prototype that was produced in the course of this project is the first attempt within the hospital to create a computerised handover and as such required the student to analyse the current processes as well as undertake a requirements analysis of the handover process. The requirements analysis was accompanied by the gathering of all paper based forms used on the ward. Together, these three aspects represented the design drivers for the project. Each of these aspects is described in more detail in the following pages.   

\subsection{Handover Overview}
Before undertaking any kind of design work, it was necessary for the student to understand and analyse the current handover processes employed on the ward.   The handover between nurses, in the formal sense, takes place three times a day, once for each shift. Handover is held at 7 am, 2 pm and 10 pm. The handover takes place in the staff lounge room on the ward. The purpose of the handover is to transfer information about the patients on the ward from one shift to the next. This is especially important in patient care because the handover process enables nurses to highlight important information about a patient that a nurse that is starting her shift \textbf{must} know in order to properly care for her assigned patients as well as to properly plan her shift. The information sources for any type of handover, not just the nurse to nurse handover, are diverse as are the communication channels employed as depicted by the figure below.
\newpage
\begin{figure}[hp]
				\centering
				\includegraphics[scale=1.0, width=90mm]{Images/Clinical_Handover_Overview}
				\caption{Clinical Handover Overview}
\end{figure} 

\noindent As the figure shows, there are five information sources that are also utilised as communication channels. Face-to-Face and verbal communication is the quickest and most efficient way of relaying information in a clinical setting. This is emphasised by the fact that clinical staff have very little time during their shift that isn't devoted to patient care. Thus, communication usually occurs ad-hoc and spur of the moment whenever staff have a minute to spare. This includes communication with doctors, other wards and other nurses. Another advantage of using face-to-face and verbal communication is the fact that clinical staff can obtain immediate feedback to their questions or information exchange. This allows staff to quickly act on information instead of wasting time waiting. This is also one of the reason why handover is being done in a verbal, usually face-to-face, manner on the ward. 
\\ \\
As stated, handover should focus on highlighting vital patient information and thus, the nurses and other clinical staff still need to go back to documented information sources to obtain a complete picture of the patient. Such documented information includes a patient's integrated notes or e-pathways as is the case on the surgical ward. A patient's integrated notes is a folder containing all the information in regard to the patient such as filled out paper based forms, doctors orders, medication orders, etc. The integrated notes thus represent everything that is known about a patient. While most wards at the SAN still use paper based integrated notes, the surgical ward has moved to an electronic system called e-Pathways, which stores all patient information electronically. In essence, e-Pathways is the electronic equivalent of a patient's integrated notes. It should also be mentioned that these integrated notes are used by all clinical staff that are charged with the care of the patient including case managers and if relevant breast navigators. This leads to the natural conclusion that it is also used as a communication medium between staff albeit a slow one. This is especially the case for any non-nursing staff as they are not on the ward at all times and the nurse caring for a patient might not be around at the time so information is written into the notes that the nurse needs to read. This in turn leads to the necessity that the nurse go through the integrated notes or e-Pathways of her patients several times a day to check for new information, something that is rather tedious and time consuming and takes the nurse away from her primary duty of patient care.
\\ \\
The last communication channel and source of information is e-mail. This channel is not used in regard to handover information but was added for completeness sake. E-mail is usually used by doctors or breast navigators between each other. It also depends on personal preference whether or not e-mail is used. 
\\ \\
As can be seen by the variety of communication channels and information sources, it is quite a difficult task to transform the current processes into an electronic form. Needless to say this would require a change, depending on each staff member this could be a drastic change, in how handover is done and how information is handled within the hospital. This will be discussed further in \ref{Difficulties}.

\subsection{Current Handover Process}
\label{Current Handover Process}
This section will outline and describe the current handover process for a nurse to nurse handover taking place during a shift change. There are two points of concern with the current handover process that needed to be addressed by the prototype. The first point of concern is the fact that currently, nurses will add ``fluff" information to their handover. This ``fluff" is any and all information that is not vital information about the patient. This fluff ranges from repeating information to mentioning things such as ``the patient is moody". This fluff is not only irrelevant information in regard to handover but it also consumes time during handover and forces the nurses of the oncoming shift to hear information that they must actively filter out.
\\ \\
The second aspect of concern is the fact that nurses on occasion only write down handover information for the patients they are responsible for. This means that the nurse is only aware of her own patients thus limiting her ability regarding other patients on the ward. A situation in which this limited knowledge of patients on the ward becomes an issue is for example when family members of patient come onto the ward and request information about the person they want to see. If a nurse does not take down handover information for that patient then she cannot be the family members the information they seek and instead has to say ``I don't know, let me get someone who does". This negatively represents the work of the nurses and that of the hospital. Another kind of situation in which the lack of patient knowledge causes problems is if emergency action must be taken for a patient. If the nurse that is responsible for the patient in distress is not available, another nurse must take her place and action the care necessary. This can only be done if the nurse is aware of the relevant patient information. While this situation is very rare it is nevertheless a possibility and should be handled appropriately. 
\\ \\
Before describing the current handover process, it is necessary to describe the two main roles that partake in the handover. The two roles are the Team Leader (TL) and the Nurses. The roles have the following responsibilities: \\ \\

\hfil\begin{tabular}{|p{7cm}|p{7cm}|}
\hline
{\hfil\bf Team Leader} & {\hfil\bf Nurse} \\
\hline
\vspace{-5mm}\begin{itemize}
\item looks after ward
\item checks to make sure registered nurses (RN) are fine
\item is also a nurse
\item calls out \gls{NFR} status to all other nurses during handover
\item in contact with the Assistant Director of Nursing (ADON)
\end{itemize} & 
\vspace{-5mm}\begin{itemize}
\item primary carer of patients
\item handover to next shift
\item administer medication
\item fill out patient forms such as Observation Chart and Nursing Care Record
\item the night shift creates roster
\end{itemize} \\ 
\hline
\end{tabular}

\newpage
\noindent When the TL comes on for duty, he or she will check each patient to see what their NFR status is. This is recorded in the patient's integrated notes, not on e-pathways. Before the start of each shift, the nurses of the on-coming shift meet in the staff room for handover. The handover usually last around 30 minutes. All of the nurses on the on-coming shift remain in the room during the entire handover. 
\\ \\ 
The first thing that happens is that the TL tells all the nurses which of the patients are NFR. Afterwards, the TL assigns each nurse a set of patients for that shift, which the nurses mark on their Patient Handover List (see Appendix \ref{Patient Handover List}). Then the nurses from the off-going shift come in, usually one by one, and verbally tell everyone in the room about the patients they were responsible for. Each nurse does her handover a little differently even though there are guidelines for handover. They tell everyone the room number, patient name, short history, diagnosis, test results, medication changes, treatment changes, drug allergies as well as outstanding test results and tests/procedures that will occur that day. They report on any variances in vital signs, bowels, or any complications that the patient has. They also mention how the patient is feeling, if they are annoyed, sleepy, disoriented and other general information such as non-critical allergies, ie. shell fish allergy. If a patient is being discharged, the nurse will say where that patient is going and if the patient has left the hospital already, he or she is stricken from the list by the on-coming nurses. 
\\ \\
The nurse who is responsible for that patient on the next shift writes down the important information on her Patient Handover List. The nurse will also create a small schedule of upcoming tests and procedures so that she can plan her shift accordingly and not be pressed for time during the shift. Any nurse can ask questions about the patient and the off-going nurse will try to answer them. If a patient is not assigned to a nurse, that nurse will not write down patient information. Should there be any doctor's notes or orders, then these are also mentioned during the handover process. When the off-going nurse is finished giving her handover she leaves the room and thus ends her shift. Should any information be missing then the on-coming nurse will check e-Pathways, the integrated notes or contact the relevant person in the hospital.
\\ \\
Below is a process diagram depicting how handover is currently undertaken based on the previous description.

\begin{figure}[hp]
				\centering
				\includegraphics[angle=-90,scale=1.0, width=120mm]{Images/Nurse-to-Nurse-Handover-Process-As-Is}
				\caption{Current Nurse-to-Nurse Handover}
\end{figure} 
\newpage

\subsection{Requirements Analysis}
Concurrently to understanding the current handover process, the student undertook a requirements analysis for handover. The student interviewed various clinical staff fulfilling varying roles within the hospital. For a complete list of roles encountered during the requirements analysis see Appendix \ref{User Roles}. The interviews allowed the student to better understand the various roles within the hospital as well as obtaining the view on handover from the major roles involved in handover. After having undertaken the interviews, the student compiled a requirements document that was then verified by staff. 
\\ \\ 
The student based his requirements model on that of the Business Analyst Body of Knowledge (\cite{IIBA}). The requirements were split into business, stakeholder, pre-requisite requirements, functional and non-functional requirements. Pre-requisite requirements were requirements that must be fulfilled in order to design a clinical handover and refer to various paper based forms obtained throughout the project.  The requirements are documented in Appendix \ref{Handover Requirements}.
\\ \\
It should be noted that the requirements gathered and documented are an attempt to fully document handover requirements and thus are not limited to requirements that can be fulfilled by iCIMS. An example where this is the case is with printing the handover information. While iCIMS has a print function, it merely prints out the web page content and does not allow for print styling. Printing out the web page handover would not aid the nursing staff in their work as each patient would reside on a separate printed page. Nurses would have to carry around seven to fifteen pages worth of patient information, which is not something that is feasible. Instead, the information system containing the handover should be able to print a handover list such as the one currently employed by the nurses.

\subsection{Form Gathering}
As part of the requirements analysis the student collected all forms in use on the ward. All forms were collected in order to identify the sources of information that ultimately is handed over between shifts. The student collected over twenty forms during the requirements analysis. After collecting all of the forms, the student went through all the forms and identified the ten most important forms in regard to handover. This was done in order to keep within the scope of the project and to reduce the number of forms to a number that could be managed by the students in regard to analysis and digitisation. Of all the forms collected, the following ten forms were deemed most important by the student in conjunction with clinical staff, in no specific order:

\begin{enumerate}
\item Nursing Care Record Form
\item Fluid Balance Chart Form
\item Intravenous Fluid Orders Form
\item Pain \& Symptom Assessment Form
\item Patient History Form
\item Adult Observation Chart
\item Medication Chart Form
\item Temporary Transfer/Handover Form
\item Patient Handover List Form
\item Patient Mobility Form
\end{enumerate}

\noindent Apart from forms 5 and 8, all forms are filled out at least once a day and thus hold the most up to date information on the patient. In order to better understand the forms and how the relevant information is documented, the student also gathered filled-in versions of these forms. A very important issue to raise here is the fact that because the forms are paper based, they are filled out by hand leading to issues with legibility. The student, on numerous occasions, encountered this problem and had to seek assistance from the relevant staff member to clarify noted information. The issue of legibility was also raised by clinical staff themselves, not only in regard to doctors but to all staff, and is something that they wished to be addressed. The filled-in forms also allowed the student to enter realistic data into the prototype in order to present the functionality to clinical staff.
\\ \\
Another issue that should be noted in regard to forms is the fact that most of the time, certain parts of forms were not filled out because they were not relevant for that patient. This presented the student with the problem that diverse variations of information was not available for all parts of forms. Unfortunately, it was unrealistic to ask clinical staff to provide such information for all possible situations because that would have required them to use a considerable amount of their time to generate the information, time that they did not have. Nevertheless, the student managed to adequately design the prototype based on the available information.

\subsection{Design Process}
As described in \ref{Overview}, the student utilised an incremental and iterative user driven development model. The realisation of the development model with iCIMS is depicted below.

\begin{figure}[hp]
				\centering
				\includegraphics[scale=1.0, width=120mm]{Images/Design-Process}
				\caption{Design Process}
\end{figure} 

\noindent The first step undertaken was to digitise the paper forms within SANSURGIMS. This meant attempting to copy the form's design and layout one to one. The reason why the forms were copied one to one was so that the student did not influence the design of the system and that the clinical staff could partake in all design steps. The one to one copy of the form also allowed the clinical staff to more easily identify the form being displayed in the system. It also allowed the student to more easily elicit staff feedback because the staff were not confused by looking at something totally new. After seeing the digital copy of the form, the clinical staff could then identify shortcomings of the forms themselves and, together with the student, undertake changes to the forms. The digitised forms were displayed in the web browser and it was based on the web page display that the clinical staff gave feedback and suggestions. 
\\ \\
This process allowed the student to incrementally add new forms into the designer and iteratively shape and mould these forms as desired with the help of the clinical staff. In order to make this more clear, Appendix \ref{Nursing Care Record Transformation} depicts the design application. It begins by showing the paper form and then showing how the same form is represented in the designer and finally how it is shown in a web browser. It should be noted that the example form has been modified to meet clinical staff needs based on design choices.
\\ \\ 
While undertaken the design steps with the clinical staff during the semester, the student was confronted with several design decisions. This stemmed partly from the requirements gathered and partly from the necessity to create a handover that was relevant and usable by numerous nursing staff. The major design decisions are outlined below.

\subsubsection{Handover by Exception}
Due to the fact that staff were highlighting the fluff issue mentioned in \ref{Current Handover Process} and the duration of handover being an issue, it was proposed that handover should be done by exception rather than by rule. This means that handover should focus on variances and abnormalities of a patient as well as information that directly affects the nurse's scheduling during her shift. This is opposed to handing over by rule in which all information about the patient is shared regardless of whether that information is critical or not. Clinical staff reasoned that a handover by exception would alleviate both the fluff issue as well as the lengthy handover issue. The student found the staff's reasoning logical and thus proceeded to create a handover by exception in which variances and abnormalities were at the centre.
\\ \\
While seaming reasonable and logical, the design decision had a major pitfall that was not discovered until user testing was undertaken. The pitfall pertains to the fact that not all nursing staff have equal experience. This means that junior nurses or student nurses will not necessarily know what is normal and what is abnormal for a patient whereas senior nurses and nursing management would, given their extensive experience. The system however must cater to both experienced and unexperienced nurses. A solution proposed by the clinical staff who noted this shortcoming was to create another handover form where all of the ``normalities" of the patient were displayed, the information that was not out of the ordinary. This would enable the experienced staff to see what they wanted to see in the handover by exception and also allow less experienced nurses to retrieve normal information on the patient in the handover displaying normalities. 

\subsubsection{Viewing a Patient at a Glance}
Clinical staff mentioned that they would like to view a patient at a glance meaning that they would see all relevant information without having to switch screens. The clinical staff wanted to move away from looking at disconnected patient information, in various screens, to seeing all of the patient information at once allowing for connections to be made between pieces of information. This design concept was raised by several clinical staff members and is reflected in the handover form creating during the course of this project. 

\newpage
\subsubsection{Minimal Scrolling}
Another design decision that affected the overall design of all forms within SANSURGIMS was the issue of scrolling. Because clinical staff were not computer literate, the situation could occur where staff would miss information simply because it was further down the screen and required scrolling down. In order to mitigate this risk, it was deemed necessary to create forms that required as little scrolling as possible. Unfortunately, scrolling could not always be avoided like it was the case with a ward patient list that was simply too long but was required to be displayed as a table. Wherever possible however, scrolling was eliminated leading to information being split into separate parts and only one part being visible at any given time. A perfect example of this design choice is reflected in the Nursing Care Record, (Appendix \ref{Nursing Care Record Transformation}). Instead of having a long scrollable form, the existing sections on the paper form were turned into separate forms within SANSURGIMS and each linked to with a button.

\subsection{Digital Handover Process}
\label{Digital Handover Process}
The culmination of the system design were the patient tracking list, patient handover and all digital forms, a total of 52 forms, as well as a new handover process. The new handover process involving SANSURGIMS is described below and following are the digital forms. It should be noted that an assumption was made in regards to data entry. As the surgical ward is already employing a e-pathways, the digital equivalent to a patient's integrated notes, it is assumed that data entry and retrieval will occur through e-pathways and that SANSURGIMS takes that data and presents it in the form of the handover.
\\ \\
During the shift, the TL will check the patient's NFR status on the ward. While doing so, the TL will update the patient information in the system. Likewise, nurses undertaking their care of patients they are responsible for will update relevant forms and information in the computer system. 
\\ \\
At the end of a shift, the nurse will go through his or her patients and ensure that all relevant information has been filled in. At the end of the shift, the TL will assign the patients on the ward to the nurses on the on-coming shift.
\\ \\
At the beginning of her shift, the on-coming nurse will log into the computer system and be presented with a list of the patients on the ward. He or she then filters that list to display only the patients that have been assigned to him or her. He or she then goes through the patient handovers one by one by clicking on the handover button for each patient on the screen. The handover screen will then display all relevant information to the nurse allowing her to obtain a picture of the patient. The relevant information includes: drains, lines, tubes, fluid input, drain volume, voiding, removal of drains/catheters, relevant doctor orders, discharge plan, patient history summary, outstanding test results, test appointments, infection risk, NFR, procedures, allergies, medications (anticoagulant, antibiotics, analgesics), fall risk, vital sign variances. 
\\ \\
Should the nurse want more information she can navigate to relevant forms in the system that contain all of the information about the patient especially the information that was not displayed during handover. This kind of information includes any non-variance information as well as peripheral information about the patient. The nurse can re-visit the handover screen for a patient at any time allowing her access to a big picture depiction of the patient.

\begin{figure}[hp]
				\centering
				\includegraphics[angle=-90,scale=1.0, width=100mm]{Images/Nurse-to-Nurse-Handover-Process-To-Be}
				\caption{Updated Nurse-to-Nurse Handover}
\end{figure} 

\newpage
\noindent The process outlined above is what handover could like like once it is based solely on a paperless system. While being a rather utopian view of the future of the clinical domain, it does reflect what \textbf{can} be done. Obviously such a drastic change in any process will entail a quite substantial change in the way staff do things during their shift. This is quite an enormous undertaking especially with end-users that are considered computer illiterate due to the nature of such users to exhibit a greater barrier to change involving computer systems. In order to reduce the barrier to change acceptance, it would be necessary to be able to print the handover information during a shift. The paper printout would act as a bridging mechanism between the purely paper based handover and the paperless handover. Indeed, clinical staff repeatedly asked about whether it would be possible to print the handover out. The printed list is viewed as a life line for the nurses and as such would need to be retained until the last possible moment. Due to constraints within the project and iCIMS, a printable version of the handover was not designed but the student did realise the importance of this information medium and as such is reflected in Appendix \ref{Handover Requirements} as part of the handover requirements.
\\ \\
Having outlined the new process, a description of the SANSURGIMS forms and workflow follows. The initial screen within SANSURGIMS is the ward patient tracking list (Appendix \ref{Patient Tracking List}). This displays all patients currently residing on the ward and includes information such as the infection control (IC), bed number, full name, age, estimated date of discharge, attending and consulting doctors as well as the attending nurse. For each patient, there is a button that will go to that patient's handover, denoted by ``Handover", and another button will take the user to the general patient information (Appendix \ref{General Patient Information}), denoted by ``i", which is where the tracking list information among other things is stored. The tracking list serves as a ward overview as well as the starting point for any actions taken for a patient.
\\ \\
Clicking on the ``Handover" button will lead the user to the handover associated with that patient. Depending on the complexity of the patient, various variances and abnormalities are displayed. Clinical staff are able to assess the complexity of the patient at a glance solely based on the amount of information displayed on the handover form. Appendix \ref{Very Uncomplex Patient} to \ref{Very Complex Patient} provide example handovers for patients ranging form very uncomplex patients to very complex patients. 


