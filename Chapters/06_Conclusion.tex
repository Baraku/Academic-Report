\section{Conclusion}
\subsection{Strengths and Weaknesses}
The strengths of the project lie in the fact that it has laid the foundation for continued development of the digital handover. The project has successfully shown the clinical staff that a digital handover would improve the way they work. As such, the project was successful as a pilot project. The handover design created has been received positively by the staff and allows for continued refinement that could lead to direct use on the ward. 
\\ \\ 
The major weakness of the project is that it only delivered a first draft of the handover. It would have been better to have gone through two or three more major versions of the handover, including the handover form containing all of the patient information for junior nurses, in order to finish the project with a more mature handover. On some level, it came as a surprise that we were wrapping up the project after the first draft, not by choice but rather because the semester ended. Being able to show a more mature handover as well as having created the second version would have left the project in a stronger position to be continued next semester.

\subsection{A Second Time}
If the project would be undertaken a second time, the student would definitely spend more time on the ward building social relationships with the clinical staff. The student would also ask for access to iCIMS from the beginning in order to understand how it works. The requirements gathering could be done during the days at the hospital and the familiarisation with iCIMS could be done on the student's own time. This would allow for issues with the tool to be identified earlier as well as allowing for more design time. A bigger focus would be given to getting \textbf{all} relevant forms into the system and evenly working on data entry forms and the handover forms so that a complete system could be shown to staff, even if down the road the data entry forms are replaced with a connection to existing SAN applications. Documentation of the project and issue and bug tracking was done quite well and the student would not change anything in that regard.

\newpage
\subsection{Future Work}
Future work on SANSURGIMS should include the creation of the normative handover, for use by junior staff, as well as the refining and adding of data entry forms. Future work should also include much more extensive usability and acceptance testing in order to adequately verify the success of the design. After having reached a significantly mature level of the nurse to nurse handover, future students could begin work on handovers including doctors, breast navigators, case managers and allied health. This work would culminate in a truly multi-disciplinary handover and reflect the overarching goal of the SAN in regards to improving handover. A further area of work that needs to be undertaken is change management in regards to handover. With the move to a computerised handover, the work processes for clinical staff would change drastically and thus must be managed. The clinical staff needs to be supported in the transition from the current paper based system to SANSURGIMS in order to ensure positive uptake and effective use of the new system. A last point of work that could be undertaken is the actual connection to current SAN applications in order to draw data from them to populate the handovers. Each of these tasks is quite an undertaking and would most likely result in them becoming sub-projects of their own.

\subsection{Concluding Remarks}
Overall, the project was a great experience. It allowed the student to work in a unique domain with great potential. It allowed the student to grow as a professional and to be part of what looks like a promising computer information system application. The student gained valuable knowledge in regards to inter-personal relationships, communication, requirements gathering and people skills in general. The project challenged the student in a way that only a real world project can. The student would do the project again, if asked, because for all its hardships it offered a unique and very valuable learning experience. 