\section{Process}
\subsection{Overview}
In order to take full advantage of the tools employed in this project as well as the opportunity to work at the hospital, a user driven development process was chosen. This allowed the student to take advantage of the tacit knowledge of clinical staff and allow the end-user to actively participate in the design of the system. The ultimate goal of any project is to supply an outcome that will be accepted by the stakeholders, in this case the end users. User driven development enabled the student to ensure that during all steps in the project, the system met stakeholder needs and requirements. The active feedback gained from clinical staff enabled the student to shape and form the system to meet user needs. It is important to note that sitting down with clinical staff will was not always possible as their priority lies with patient care. In order to facilitate a continued design process throughout the semester a number of methods were employed. The student utilised participative observations, interviews as well as feedback notes to gain an adequate understanding of the user needs and requirements.

\subsubsection{Participative Observations}
As this was the first time the student undertook a IT project within the health domain, it was important to learn how clinical staff worked. Participative observations facilitated this need through \emph{shadowing} clinical staff during various points in their shift. This included sitting in on clinical handover, the process through which the nurse of the previous shift conveys patient information to the nurses of the oncoming shift. By sitting in on handover, the student was able to gain an understanding of the overall process of clinical handover as well as experience the shortcomings and inefficiencies first hand. This enabled the student to relate to the clinical staff when discussing the system design. 
The student did not actively participate in handover, that is to say that the student merely observed and did not give handover. Thus, the student acted like a `fly on the wall', remaining in the background looking on. This method proved vital to the project as it provided a first hand experience of the current situation. One disadvantage of using participative observations is that the student was overwhelmed at first by the sheer amount of information that he was exposed to. It took some time and effort on the part of the student to sift through, filter and understand the information in order to make the most efficient use of it. As the project progressed, this information overload continually decreased.

\subsubsection{User Interviews}
In order to supplement participative observation, user interviews were undertaken with various clinical staff fulfilling different roles within the hospital. This enabled the student to understand the viewpoints of various staff in regards to what they do as well as their view on clinical handover. The interviews lasted between ten and fifteen minutes and were held casually. During the interviews, the student asked specific questions about handover and the user's involvement. The interviews were held on the ward or in offices depending on the role of the staff being interviewed. The interviews also served the purpose of building a relationship with the clinical staff. The student's aim was to go back to the interviewed staff throughout the project and attain their feedback on the system design. This social component was very important in order to elicit staff support especially nursing staff as their time was extremely limited. Not undertaking this social component would have made the work of the student much more difficult.

\subsubsection{Feedback Notes}
During the course of the project, the student met with clinical staff at various times in order to show them the current status of the design as well as to get answers to open questions. During each of these encounters, notes were taken in order to preserve the exchange allowing the student to return to the notes at a  later time to refresh his memory. In order to maximise the use of time with the clinical staff, the student was accompanied by another student who assisted with taking notes during the feedback sessions. This allowed the student to focus on the staff and not waste time trying to write down information as well as facilitating a smooth flow during the meetings with staff.


\subsection{Tools and Skills}
From the beginning of the project, one of the key aspects was the use of the \gls{iCIMS}.  Having been created at the University of Sydney recently, the project was meant to trial the tool in a real world scenario. The main ideology behind iCIMS is to use a graphic interface to build a system thus avoiding the necessity for programming skills. This means that end-users could actively participate in designing the system by creating the system together with the designer. iCIMS aims to be applicable in any clinical situation and thus utilises Ockham's Razor of Design (\cite{Budd}), which states that :

\begin{quote}
\center\emph{``a design should use the minimal number of entities \\ with their maximal generalisations"}
\end{quote}

In conjunction with using iCIMS, the student utilised Trac, a bug and issue tracking system that also offers documentation functionality in form of a wiki. Two separate instances of Trac were used because the iCIMS project had an existing instance. All bugs and issues were documented in the iCIMS instance of Trac and all project documentation was created in a separate instance of Trac available for the project. All relevant project information, such as interview notes, process flows, etc., was documented on the Trac instance. Apart from playing a vital role in the completion of the project academically, the wiki will also serve as a starting point for future students continuing the project. This ensures that the student's efforts were not in vain and will allow future students to more easily find their way while undertaking future projects.
\\ \\
Apart from electronic tools, the student needed to use communication and people skills in order to successfully undertake this project. Communication and people skills played a vital role as the student was dealing with end-users that were very technology-adverse as well having a low computer-literacy. Inter-personal skills allowed the student to found common ground and terminology with which to communicate with end-user. It also allowed the student to quell most reservations that clinical staff had towards the project or its purpose.

\subsection{Scope and Schedule}
