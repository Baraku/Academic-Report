
%%%%%%%%%%%%%%%%%%%
% Latex Headers for Bakkarbeit.tex
%%%%%%%%%%%%%%%%%%%


\documentclass [12pt, a4paper, bibtotocnumbered, liststotocnumbered, normalheadings]{scrartcl}
\usepackage[latin9]{inputenc}
\usepackage[english]{babel}

% Graphiken einbinden
\usepackage{graphicx}
\usepackage{float}

% Anklickbare Links im PDF file
\usepackage{url}
\usepackage[pdftex, bookmarks, bookmarksopen=true, bookmarksnumbered=true]{hyperref}

% Glossary
\usepackage{glossaries}
\makeglossaries

\newglossaryentry{PBTM}{name={PBTM},description={\textbf{P}ay \textbf{B}y \textbf{T}he \textbf{M}onth. The acronym used to describe this project}}

% PDF Titel, Autor etc setzen:
\hypersetup{
	pdftitle={Pay By The Month Application Rewrite},
	pdfauthor={Patrick Mumme},
}

% Seite einrichten
\usepackage{geometry}
\geometry{verbose, a4paper, tmargin=2cm, bmargin=1.7cm, lmargin=2cm, rmargin=2cm, headsep=0cm}

% Zeilenabstand
\usepackage{setspace}
\setstretch{1.3}

% Kopf- und Fusszeilen
\usepackage{fancyhdr}
\pagestyle{fancy}
\headheight33pt
\headsep1cm
\footskip1cm
\textheight23cm

\lhead{\includegraphics[height=1cm]{Images/usyd_logo}}
\chead{} 
\rhead{\nouppercase{\leftmark}} 

\lfoot{PBTM} 
\cfoot{} 
\rfoot{Page \thepage} 

% Zitate aus Bibtex Bibliographien
%\usepackage[square, authoryear, sort]{natbib}
% \bibliographystyle{plainnat}
\usepackage[square, authoryear]{natbib}
\bibliographystyle{apa}

%\usepackage[style=apa]{natlib}
%\usepackage{csquotes}

% Tabellen 
\usepackage{tabularx} % fuer einseitige tabellen
\usepackage{supertabular} % mehrseitige tabellen

% Source Code Listings
\usepackage{listings} 
\lstset{
  float,
  basicstyle=\small, 
  tabsize=2, 
  numbers=left, 
  numberstyle=\tiny, 
  numbersep=5pt, 
  frame=lines, 
  breaklines=false,
  prebreak={\mbox{\ensuremath{\hookleftarrow}}},
  postbreak={\space\space},
  breakindent=0pt,
  captionpos=b
}


% Kommentare
% Damit kann man \begin{comment}...\end{comment} verwenden!
\usepackage{comment}
\usepackage{fancybox}
\newenvironment{commentenvironment}% 
{\begin{Sbox}\begin{minipage}}% 
{\end{minipage}\end{Sbox}\shadowbox{\TheSbox}} 
\specialcomment{comment}
{\begin{commentenvironment}{\textwidth}}
{\end{commentenvironment}}

% Alle Kommentare ausblenden geht so:
%\excludecomment{comment}


% Aufzaehlungen mit enumerate
\usepackage{enumerate}


% Lyx Listen f�r Kompatibilit�t mit Lyx Quellcode
\newenvironment{lyxlist}[1]
{\begin{list}{}
{\settowidth{\labelwidth}{#1}
 \setlength{\leftmargin}{\labelwidth}
 \addtolength{\leftmargin}{\labelsep}
 \renewcommand{\makelabel}[1]{##1\hfil}}}
{\end{list}}


% Verhinderung von "Schusterjungen"
% einzelne Absatzzeilen auf der Seite unten
\clubpenalty = 10000

% Verhinderung von "Hurenkindern"
% einzelne Zeilen eines Absatzes am Kopf der Seite
\widowpenalty = 10000



